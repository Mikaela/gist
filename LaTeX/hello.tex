% I think this was mandatory, paper size, font size and type
% artikel3 is NL, but more pleasant to European than article
% Potentially interesting options:
% * onecolumn (default) or twocolumn
% * final (default) or draft
\documentclass[a4paper,12pt]{artikel3}
% I am unsure on this being mandatory, but explicit is probably better than
% implicit?
%\usepackage[T1]{fontenc}
% Definitely reduntant, this library book spoke of latin1. Don't uncomment
% with cmap
%\usepackage[utf8]{inputenc}
% I can specify multiple languages for hyphenation! texlive-babel
\usepackage[esperanto,english,finnish]{babel}
% Noto Serif font. Remember to install texlive-fontaxes, texlive-noto
\usepackage{noto-serif}
% texlive-cmap makes pdflatex copy-pasteable
\usepackage{cmap}
% This package is great and I want to be reminded of it whenever I use \LaTeX
\usepackage{censor}

% Uncomment to uncensor in output
%\StopCensoring

\title{Saluton Mondo!}
\author{Aminda Suomalainen}
\date{\today}


% Text begins here
\begin{document}
\maketitle

\section*{Hello World!}

Hello Wörld! This is a simple \LaTeX\ template file or similar where I can
cheat how does this work again. Possibly with even comments!

\section*{Samahko suomeksi!}

Hei maailma, heitän hiukan suomenkieltäkin tähän asiakirjaan perustellakseni
ääkkösten käyttöä!

\section*{Playing with censor}

\xblackout{If the source is private, this is a secret!

So don't go talking about it!}

\def \Password{\censor*{8}}
\def \Huippusalainen{\censor*{14}}

Salasana on \Password. Älä unohda salasanan olevan \Password. Tämä tieto on \Huippusalainen.

% And ends here
\end{document}
