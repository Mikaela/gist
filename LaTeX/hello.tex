% Zeroth rule: when using git and \LaTeX, one sentence equals one line, forget staying pretty in plaintext editor!
% To have paragraphs, you will leave an empty line in the middle anyway.
%
% I think this was mandatory, paper size, font size and type artikel3 is NL, but more pleasant to European than article
%
% Potentially interesting options:
% * onecolumn (default) or twocolumn
% * final (default) or draft
% * font size can be defined here after a4paper e.g. 12pt
\documentclass[a4paper]{artikel3}
% I can specify multiple languages for hyphenation! texlive-babel
% The last is main document language
\usepackage[english,finnish,esperanto]{babel}
% Noto Serif font. Remember to install texlive-fontaxes, texlive-noto
\usepackage{noto-serif}
% texlive-cmap makes pdflatex copy-pasteable
\usepackage{cmap}
% This package is great and I want to be reminded of it whenever I use \LaTeX
\usepackage{censor}

% Uncomment to uncensor in output
%\StopCensoring

\title{Saluton Mondo!}
\author{Aminda Suomalainen}
\date{\today}


% Text begins here
\begin{document}
\maketitle

\selectlanguage{english}

\section*{Hello World!}

Hello Wörld! This is a simple \LaTeX\ template file or similar where I can cheat how does this work again.
Possibly with even comments!

\selectlanguage{finnish}

\section*{Samahko suomeksi!}

Hei maailma, heitän hiukan suomenkieltäkin tähän asiakirjaan perustellakseni ääkkösten käyttöä!

\selectlanguage{english}

\section*{Playing with censor}

\xblackout{If the source is private, this is a secret!

So don't go talking about it!}

\def \Password{\censor*{8}}
\selectlanguage{finnish}
\def \Huippusalainen{\censor*{14}}

Salasana on \Password.
Älä unohda salasanan olevan \Password.
Tämä tieto on \Huippusalainen.

\selectlanguage{english}

\section*{Remember!}

To have a non-breaking space use \textasciitilde\ (a tilde).
A backslash would be \textbackslash.
A forced linechange is \textbackslash\textbackslash.

If you were to censor things using \textbackslash xblackout containing unicode, the unicode chars would need to be within curly brackets.

\selectlanguage{finnish}

\xblackout{T{ä}h{ä}n tapaan!}
Kiva suomenkielinen sana on \xblackout{h{ää}y{ö}aie}!

% And ends here
\end{document}
